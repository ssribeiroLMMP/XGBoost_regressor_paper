\section{Introduction}
\label{sec:Introduction}
% The oil and gas industry has a huge relevance in people's life. This business surrounds humans in the means of locomotion, as a thermal source, providing raw material for several other industries. Due to its importance, its production has great relevance in the world economy.

The oil and gas industry provides fuel for locomotion and thermal sources, it also provides raw materials for several other industries.
Due to its relevance in human life, its production has great relevance in the world economy.



In addition to applications for phase identification, AI techniques can also be found in forecasting works such as Nait Amar and Zerabi\cite{nait2020combined}, in which the downhole pressure prediction was made, and there are also works forecast of hydrocarbon production (\cite{cao2016data}, \cite{zanjani2020data}).

Aiming precisely to facilitate future production planning, this work used a machine learning model to learn the relationship between the volumes of each phase produced and other information from sensors coupled to the well. With a properly trained model, we are able to make predictions of the production volume of each phase produced based on information from the sensors. The model used was $extreme gradient boost$ (XGBoost)\cite{chen2016xgboost} which is a machine learning model with supervised learning, which has been applied by several works to make predictions in different areas such as sales forecasts (\cite{shilong2021machine} ) and oil price forecasts ( \cite{gumus2017crude}).

This work structure starts with a brief explanation of the dataset used with the respective considerations about the variables used to build the model, then the methodology used to make the model, followed by the results, and finally the conclusions of the work.