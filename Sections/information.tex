\section{Informations}
\label{sec:Informations}

\textbf{XGBoost: A Scalable Tree Boosting System}

Tree boosting is a highly effective and widely used machine learning method. In this paper, we describe a scalable end-to-end tree boosting system called XGBoost, which is used widely by data scientists to achieve state-of-the-art results on many machine learning challenges. We propose a novel sparsity-aware algorithm for sparse data and weighted quantile sketch for approximate tree learning. More importantly, we provide insights on cache access patterns, data compression and sharding to build a scalable tree boosting system. By combining these insights, XGBoost scales beyond billions of examples using far fewer resources than existing system


\\

\textbf{A Simple and Fast Baseline for Tuning Large XGBoost Models}


XGBoost, a scalable tree-boosting algorithm, has proven effective for many prediction tasks of practical interest, especially using tabular datasets. Hyperparameter tuning can further improve the predictive performance, but unlike neural networks, full-batch training of many models on large datasets can be time consuming. Owing to the discovery that (i) there is a strong linear relation between dataset
size & training time, (ii) XGBoost models satisfy the ranking hypothesis, and (iii) lower-fidelity models can discover promising hyperparameter configurations, we show that uniform subsampling makes for a simple yet fast baseline to speed up the tuning of large XGBoost models using multi-fidelity hyperparameter optimization with data subsets as the fidelity dimension. We demonstrate the effectiveness
of this baseline on large-scale tabular datasets ranging from 15 − 70GB in size.

\\

\textbf{Solving problems of the oil and gas sector using machine learning algorithms}

The article describes the tasks of the oil and gas sector that can be
solved by machine learning algorithms. These tasks include the
study of the interference of wells, the classification of wells
according to their technological and geophysical characteristics, the
assessment of the effectiveness of ongoing and planned geological
and technical measures, the forecast of oil production for individual
wells and the total oil production for a group of wells, the forecast
of the base level of oil production, the forecast of reservoir
pressures and mapping. For each task, the features of building
machine learning models and examples of input data are described.
All of the above tasks are related to regression or classification
problems. Of particular interest is the issue of well placement
optimisation. Such a task cannot be directly solved using a single
neural network. It can be attributed to the problems of optimal
control theory, which are usually solved using dynamic
programming methods. A paper is considered where field
management and well placement are based on a reinforcement
learning algorithm with Markov chains and Bellman's optimality
equation. The disadvantages of the proposed approach are revealed.
To eliminate them, a new approach of reinforcement learning based
on the Alpha Zero algorithm is proposed. This algorithm is best
known in the field of gaming artificial intelligence, beating the
world champions in chess and Go. It combines the properties of
dynamic and stochastic programming. The article discusses in detail
the principle of operation of the algorithm and identifies common
features that make it possible to consider this algorithm as a
possible promising solution for the problem of optimising the
placement of a grid of wells. 

\\

\textbf{Performing Predictive Analysis using Machine Learning on the Information Retrieved from Production Data of Oil & Gas Upstream Segment}

Abstract—Machine learning is an area of knowledge, which
supports many of the established and reliable techniques in
Artificial intelligence. Oil and gas industry involve many sensors
to collect data continuously. Especially the main focus, is on the
Production data which will help the industry to perform
Predictive analysis that will forecast what outputs we may get in
future. The current research work focuses on the data produced
from an oil well, over a month and then tries to predict the
average oil rate, based on certain elements. In order to perform
this, a predictive tool RapidMiner is used, and Regression model
is applied. This research work helps in predicting the most
dependent factor on the predictive variable, which is Average Oil
Rate. 

\\
\textbf{Crude oil price forecasting using XGBoost}

Abstract—One of the most important role of economic
variables in today's world countries are the price and the change
of the price of crude oil. Changes in the price of crude oil have a
very critical role in terms of treasury and budget, both in
company and state planning. For example, one may choose one of
the energy or natural gas indexed energy production plans based
on the trend of the crude oil price, for planning to meet the need
for electricity next year. Accurate forecasting of the crude oil
price and realization of the forecasts based on this forecast will
provide savings or gains in government and corporate economies,
which can reach billions of dollars. There is a great need for this
estimation in countries where crude oil production is low and
heavily dependent on crude oil import. In this paper, the
parameters which are the factors affecting the crude oil prices
will be interpreted using XGBoost, a gradient boosting model,
from machine learning libraries and estimation will be made.

\\
\textbf{Well Performance Classification and Prediction: Deep Learning and Machine Learning Long Term Regression Experiments on Oil, Gas, and Water Production}

In the oil and gas industries, predicting and classifying oil and gas production for hydrocarbon wells is difficult. Most oil and gas companies use reservoir simulation software to predict future
oil and gas production and devise optimum field development plans. However, this process costs an
immense number of resources and is time consuming. Each reservoir prediction experiment needs
tens or hundreds of simulation runs, taking several hours or days to finish. In this paper, we attempt
to overcome these issues by creating machine learning and deep learning models to expedite the
process of forecasting oil and gas production. The dataset was provided by the leading oil producer,
Saudi Aramco. Our approach reduced the time costs to a worst-case of a few minutes. Our study
covered eight different ML and DL experiments and achieved its most outstanding R2 scores of 0.96
for XGBoost, 0.97 for ANN, and 0.98 for RNN over the other experiments.

\\

\textbf{Application of Machine Learning Method of Data-Driven Deep Learning Model to Predict Well Production Rate in the Shale Gas Reservoirs}

Reservoir modeling to predict shale reservoir productivity is considerably uncertain and
time consuming. Since we need to simulate the physical phenomenon of multi-stage hydraulic
fracturing. To overcome these limitations, this paper presents an alternative proxy model based
on data-driven deep learning model. Furthermore, this study not only proposes the development
process of a proxy model, but also verifies using field data for 1239 horizontal wells from the Montney
shale formation in Alberta, Canada. A deep neural network (DNN) based on multi-layer perceptron
was applied to predict the cumulative gas production as the dependent variable. The independent
variable is largely divided into four types: well information, completion and hydraulic fracturing
and production data. It was found that the prediction performance was better when using a principal
component with a cumulative contribution of 85% using principal component analysis that extracts
important information from multivariate data, and when predicting with a DNN model using 6
variables calculated through variable importance analysis. Hence, to develop a reliable deep learning
model, sensitivity analysis of hyperparameters was performed to determine one-hot encoding,
dropout, activation function, learning rate, hidden layer number and neuron number. As a result, the
best prediction of the mean absolute percentage error of the cumulative gas production improved
to at least 0.2% and up to 9.1%. The novel approach of this study can also be applied to other shale
formations. Furthermore, a useful guide for economic analysis and future development plans of
nearby reservoirs

\\

\textbf{Modified aquila optimizer for forecasting oil production}


Oil production estimation plays a critical role in economic plans for local governments and organizations. Therefore, many studies applied different Artificial Intelligence (AI) based methods to estimate oil production in different countries. The Adaptive Neuro-Fuzzy Inference System (ANFIS) is a well-known model that has been successfully employed in various applications, including time-series forecasting. However, the ANFIS model faces critical shortcomings in its parameters during the configuration process. From this point, this paper works to solve the drawbacks of the ANFIS by optimizing ANFIS parameters using a modified Aquila Optimizer (AO) with the Opposition-Based Learning (OBL) technique. The main idea of the developed model, AOOBL-ANFIS, is to enhance the search process of the AO and use the AOOBL to boost the performance of the ANFIS. The proposed model is evaluated using real-world oil production datasets collected from different oilfields using several performance metrics, including Root Mean Square Error (RMSE), Mean Absolute Error (MAE), coefficient of determination (R2), Standard Deviation (Std), and computational time. Moreover, the AOOBL-ANFIS model is compared to several modified ANFIS models include Particle Swarm Optimization (PSO)-ANFIS, Grey Wolf Optimizer (GWO)-ANFIS, Sine Cosine Algorithm (SCA)-ANFIS, Slime Mold Algorithm (SMA)-ANFIS, and Genetic Algorithm (GA)-ANFIS, respectively. Additionally, it is compared to well-known time series forecasting methods, namely, Autoregressive Integrated Moving Average (ARIMA), Long Short-Term Memory (LSTM), Seasonal Autoregressive Integrated Moving Average (SARIMA), and Neural Network (NN). The outcomes verified the high performance of the AOOBL-ANFIS, which outperformed the classic ANFIS model and the compared models.

\\

\textbf{Forecasting oil production using ensemble empirical model decomposition based Long Short-Term Memory neural network}

Oil production forecasting is an important means of understanding and effectively developing reservoirs. Reservoir numerical simulation is the most mature and effective method for production forecasting, but its accuracy mostly depends on high-quality history matching and accurate geological models. In order to achieve fast and accurate production predicting, an ensemble empirical mode decomposition (EEMD) based Long Short-Term Memory (LSTM) learning paradigm is proposed for oil production forecasting. In this paper, the original oil production series are first split into training set and test set. The data of test set is gradually added to the training set and decomposed by EEMD to obtain multiple intrinsic mode functions (IMFs). The stability of IMFs is analyzed by its Means and curve similarity computed by Dynamic time warping (DTW). Then proper number of stable IMFs are selected as predictor variables for machine learning. Considering the variation trend and context information of production series, LSTM is utilized to establish predictive model for production forecasting. The optimal hyper-parameters of LSTM are determined by Genetic algorithm (GA). For evaluation and verification purpose, the proposed model is applied to two actual oilfields from China. Empirical results demonstrated that the proposed approach is capable of giving almost perfect production forecasting.



\\

\textbf{A comparative machine learning study for time series oil production forecasting: ARIMA, LSTM, and Prophet}

It is challenging to predict the production performance of unconventional reservoirs because of the sediment heterogeneity, intricate flow channels, and complex fluid phase behavior. The traditional oil production prediction methods (e.g., decline curve analysis and reservoir simulation modeling forecasting) are subjective. This paper presents a machine learning-based time series forecasting method, which considers the existing data as time series and extracts the salient characteristics of historical data to predict values of a future time sequence. We used time series forecasting because of the historical fluctuations in production well and reservoir operations. Three algorithms were studied and compared to address the limitations of traditional production forecasting: Auto-Regressive Integrated Moving Averages (ARIMA), Long-Short-Term Memory (LSTM) network, and Prophet. This study starts with the representative oil production data from a well located in an unconventional reservoir in the Denver-Julesburg (DJ) Basin. 70% of the data was used for model training, whereas the remaining 30% of data was used to evaluate the performance of the above-mentioned methods. Then, the decline curve analysis and reservoir simulation modeling forecasting were applied for comparison. The advantages of the machine-learning models include a simple workflow, no prior assumption about the reservoir type, fast prediction, and reliable performance prediction for a typical fluctuating declining curve. More importantly, the ‘Prophet’ model captures production fluctuation caused by winter impact, which can attract the operator's attention and prevent potential failures. This has rarely been explored and discussed by previous studies. The application of ARIMA, LSTM, and Prophet methods to 65 wells in the DJ Basin show that ARIMA and LSTM perform better than Prophet—probably because not all oil production data include seasonal influences. Furthermore, the wells in the nearby pads can be studied using the same parameter values in ARIMA and LSTM for predicting oil prediction in a transferred learning framework. Specifically, we observed that ARIMA is robust in predicting the oil production rate of wells across the DJ Basin.